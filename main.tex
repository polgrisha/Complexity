\documentclass[a4paper,14pt,russian]{article}

\usepackage[utf8]{inputenc}
\usepackage[T2A]{fontenc}
\usepackage[russian, english]{babel}
  
\usepackage{graphicx} 
\usepackage{amssymb,amsfonts,amsmath,amsthm} 
\usepackage{indentfirst} 
\usepackage[usenames,dvipsnames]{color} 
\usepackage{makecell}
\usepackage{multirow} 
\usepackage{ulem} 
\usepackage{soul}
\usepackage{booktabs}
\usepackage{graphicx}
\DeclareGraphicsExtensions{.pdf,.png,.jpg,.eps}
 \usepackage{caption}

\linespread{1.3} 
\frenchspacing

\linespread{1.5}


\topmargin=-1in
\headheight=13mm
\headsep=7mm
\footskip=2cm

\textheight=254mm

\oddsidemargin=-5mm
\evensidemargin=-5mm
\marginparsep=2cm
\marginparwidth=0pt


\textwidth=170mm


\begin{document}

\begin{center}
\hfill \break
\large{МОСКОВСКИЙ ФИЗИКО-ТЕХНИЧЕСКИЙ ИНСТИТУТ}\\
\hfill \break
\normalsize{факультет инноваций и высоких технологий}\\
 \hfill \break
\normalsize{Кафедра дискретной математики}\\
\hfill\break
\hfill \break
\hfill \break
\hfill \break
\vfill
\Large{Задача упаковки по корзинам}\\
\hfill \break
\hfill \break
\large{Индивидуальный проект}

\hfill \break
\hfill \break

\end{center}
 
\hfill \break
\vfill

\hfill
\large{Поляков Григорий 799}
\break

\thispagestyle{empty} 
 
 
\newpage
     
    \tableofcontents
    
\newpage
 
\newpage
\section{Введение}

{\large \bf Постановка задачи}

Рассмотрим некоторое множество элементов под номерами $I = \{1, ..., n\}$. Каждый элемент $i \in I$ имеет "размер" $a_i \in (0, 1]$. Пусть также дано множество так называемых корзин $B = {1, ..., n}$. 

Требуется распределить в все элементы по наименьшему числу корзин так, чтобы сумма размеров всех элементов в каждой корзине не превосходила 1.

Более формально, построить такую функцию $\xi : I \rightarrow B$, что: $$\forall b \in B \:\: \sum_{i \in I \: : \xi(i) = b} s_i \leq 1$$

\newpage
\section {Задача о $o(\frac{3}{2} - \varepsilon)$ - приближении}

{\bf Утверждение:} \textit{Задача {\bf NP}-трудна}

\textit{Доказательство:} Сведение задачи PARTITION.

Для начала сформулируем задачу PARTITION. Дан конечный набор целых чисел $A$. Определить существует ли подмножество $B \subset A$, что $\sum_{a \in B} a = \sum_{a \in {A \textbackslash B}} a$

Из курса знаем, что PARTITION является {\bf NP} -полной. 

Построим полиномиально-вычислимую функцию сведения. Пусть задан набор чисел $a_1, ..., a_n$, для которых нужно решить задачу  PARTITION. Пусть $S_a = \sum_{i=1}^n a_i$. Тогда $s_i = f(a_i) = \frac{2a_i}{S_a}$. Покажем, что существует подмножество для PARTITION $\Leftrightarrow$ минимальное количество корзин для разбиения $i_1, ..., i_n$ c весами $s_1, ..., s_n$ равно $2$.

Обозначим $OPT$ - решение задачи разбиения на корзины.

\begin{itemize}
  \item $\Rightarrow$ Заметим, что $OPT$ > 1, т.к. $\sum_{i = 1}^n s_i = 2$. Если существуют подмножества $B$ и $B \textbackslash A$ - решения задачи PARTITION, то существует разбиение набора $s_1, ..., s_n$ на две корзины, сумма в каждой равна единице.
  
  \item $\Leftarrow$ Пусть $OPT$ = 2. Тогда сумма весов в каждой корзине равна 1. Следовательно, подмножества исходного набора $a_1, ..., a_n$, соответствующие разбиению на корзины, и будут решением задачи PARTITION.
  
\end{itemize}

В данном сведении приближение не играет роли, т.к. если решение задачи разбиения на корзины равно 2. То алгоритм, решающий задачу о $o(\frac{3}{2} - \varepsilon)$ - приближении всегда его найдёт, ведь $2 \cdot \frac{3}{2} = 3$, следовательно ответ 3 - уже будет худше, чем $(\frac{3}{2} - \varepsilon)$ - приближение.

\newpage
\section {Алгоритм, дающий $(1 + \varepsilon)OPT + 1$ корзин}

{\bf Теорема:} \textit{Пусть $OPT$ оптимальное число корзин. Тогда, для любого $\varepsilon \in (0, 1]$ существует полиномиальный по $n$ алгоритм, дающий разбиение на $(1 + \varepsilon) \cdot OPT + 1$ корзин в худшем случае.}

{\bf Лемма 1:} \textit{Пусть $\varepsilon > 0$ и $d > 0$ - некоторые константы. Пусть также $\forall i \in I: s_i \geq \varepsilon$ и количество различных среди всех $s_i$ не превосходит $d$. Тогда, существует полиномиальный по n алгоритм, дающий в точности оптимальное значение для задачи упаковки по корзинам}

\textit{Доказательство:} Проведём полный перебор всех возможных упаковок. Максимальное число элементов в одной корзине равно $\lfloor{\frac{1}{\varepsilon}\rfloor}$. Обозначиим это число за m. 

Тогда, макимальное количество способов заполнить одну корзину $C_{m+d-1}^{m}$(в соответствии с формулой количества сочетаний с повторениями). Тогда, количество способов заполнить все n корзин не превосходит $n \cdot C_{n+m-1}^{m}$. Последнее выражение полиномиально зависит от n.

Таким образом, алгоритм просто будет перебирать все возможные варианты расстановки.
$\qedsymbol$

{\bf Лемма 2:} \textit{Пусть $\forall i \in I: s_i \geq \varepsilon$. Тогда, существует полиномиальный по n алгоритм, дающий разбиение на $(1 + \varepsilon)OPT$ корзин , где $OPT$ - оптимальное число корзин}

\textit{Доказательство:} (Во всех случаях мы сравниваем элементы по их рамеру)

Отсортируем элементы для разбинениия по увеличению размера $s_i$. Полученный набор назовём $Q$. Затем разделим их на $P + 1$ группу. Тогда в каждой группе будет не более $\lfloor{n/P\rfloor}$ элементов. Проведём разбиение так, чтобы во всех группах, кроме последней элементтов было поровну. Из полученного отсортированного набора элементов построим два новых набора $H$ и $J$ следующим образом: 

\begin{itemize}
  \item $L:$ набор $Q$, в котором размер $s_{(i)}$ каждого элемента заменён на размер наименьшего по размеру элемента в соответствующей ему группе.
  \item $G:$ набор $Q$, в котором размер $s_{(i)}$ каждого элемента заменён на размер наибольшего по размеру элемента в соответствующей ему группе.
\end{itemize}

Тогда обозначим $OPT(L)$, $OPT(Q)$, $OPT(G)$ - оптимальные решения задачи упаковки по корзинам для наборов $Q$, $H$ и $P$ соответственно.

Очевидно, что:
$$OPT(L) \leq OPT(Q) \leq OPT(G)$$

Покажем также, что $OPT(G) \leq OPT(L) + \lfloor{n/P\rfloor}$. Пусть мы решили задачу упаковки для $L$, получив некоторую функцию соответствия $\xi$. Построим новую функцию $\psi$. Заметим, что все элементы в группе $i+1$ набора $L$ не меньше элементов из группы $i$ набора $G$(по построению). Тогда $\psi((i, l)) = \xi((i+1, l))$, где $i$ - номер группы, $l$ - номер элемента в группе. Элементы последней группы положим в ещё не заполненные корзины(каждый элемент в отдельную корзину). Всего элементов в последней группе $\leq \lfloor{n/P\rfloor}$, откуда и следует, что $OPT(G) \leq OPT(L) + \lfloor{n/P\rfloor}$.

Тогда:
$$ OPT(G) \leq OPT(L) + \lfloor{n/P\rfloor} \leq OPT(Q) + \lfloor{n/P\rfloor}$$

Подберём такое P, что $\lfloor{n/P\rfloor} \leq \varepsilon \cdot OPT(Q)$.
Тогда, $OPT(G) \leq (1 + \varepsilon) \cdot OPT(Q)$. 

Из \textit{леммы 1} следует, что существует алгоритм для точного решения задачи упаковки для набора G. Заметим, что по построению набора $G$ (все элементы в нём не меньше элементов из исходного набора $Q$), полученное данным алогритмом соответствие подойдёт и для набора $Q$.
$\qedsymbol$

\textit{Доказательство Теоремы:} Разобьём исходный набор элементов $I$ на два непересекающихся поднабора:

\begin{itemize}
  \item набор $I'$, в котором для любого $i \in I'$ размер $s_{i} \geq \varepsilon$ 
  \item набор $I''$, в котором для любого $i \in I''$ размер $s_{i} < \varepsilon$ 
\end{itemize}

Рассмотрим $\delta = \varepsilon / 2 < 1/2$

Для набора $I'$ и $\delta$ применима \textit{Лемма 2}. Запустим алгоритм, полученный в \textit{Лемме 2} на наборе элементов $I'$. Получим некоторую упаковку(соответствие). 

Затем, для оставшихся элементов из $I''$ и полученной упаковки применим жадный алгоритм. Отсортируем из по убыванию размера $s_i$, затем для каждого элемента в порядке сортировки будем класть его в первую корзину, для которой сумма размеров уже лежащих элементов вместе с его размером не превзойдёт 1.

Обозначим полученное количество корзин за B.

Если после приминения жадного алгоритма новых корзин не добавилось, то из \textit{Леммы 2} получаем, что  
$$ B \leq (1 + \delta)OPT \leq (1 + \epsilon)OPT + 1$$

Иначе, получим некоторую упаковку по корзинам, где для каждой (кроме может быть одной) заполненной нами корзины её остаточная вместимость ($1 - \sum_{i: \xi(i) = b_j} s_i$) не превосходит $\delta$. 

Тогда, 
$$OPT \geq \sum_{i \in I} s_i \geq (B - 1)(1 - \delta)$$

Используя неравнество $\frac{1}{1 - x} \leq 1 + 2x \: (\forall x \in [0, 1/2])$, получаем:

$$ B \leq \frac{OPT}{1 - \delta} + 1 \leq (1 + 2\delta) \cdot OPT + 1 =  (1 + \varepsilon) \cdot OPT + 1 $$ 
$\qedsymbol$
  
  
\newpage

\addcontentsline{toc}{section}{Список литературы}
\bibliographystyle{plain}
\bibliography{lit}
\renewcommand\refname{\centering Список литературы}
\begin{thebibliography}{9}

\bibitem{book} 

Fernandez de la Vega, W., & Lueker, G. (2007). Bin packing can be solved within 1 + $\varepsilon$ in linear time. Combinatorica, 1(4), 349-355.

\bibitem{link1}

\begin{verbatim}
http://u.cs.biu.ac.il/~amir/bin-packing.pdf
\end{verbatim}

\bibitem{link2}

\begin{verbatim}
http://ac.informatik.uni-freiburg.de/lak_teaching/ws11_12/
combopt/notes/bin_packing.pdf
\end{verbatim}
   
\end{thebibliography}

\end{document}